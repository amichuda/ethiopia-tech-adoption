\documentclass{beamer}
\usepackage[utf8]{inputenc}

%\usepackage{utopia} %font utopia imported

\usepackage{amsthm, booktabs, setspace, natbib, amsfonts,amssymb,epstopdf,textcomp,listings, xcolor}

\usepackage{adjustbox} % Shrink stuff
\usepackage{pgfpages}


\usepackage{tikz}
\usepackage{verbatim}
\setbeamertemplate{note page}{\pagecolor{yellow!5}\insertnote}
\usetikzlibrary{positioning}
\usetikzlibrary{snakes}
\usetikzlibrary{calc}
\usetikzlibrary{arrows}
\usetikzlibrary{decorations.markings}
\usetikzlibrary{shapes.misc}
\usetikzlibrary{matrix,shapes,arrows,fit,tikzmark}

\usepackage{amsmath}
\usepackage{mathpazo}
\usepackage{hyperref}
\usepackage{lipsum}
\usepackage{multimedia}
\usepackage{graphicx}
\usepackage{multirow}
\usepackage{dcolumn}
\usepackage{bbm}
\usepackage[space]{grffile}
\usepackage{tabularx}
\usepackage{subfig}
\usepackage{float}
\usepackage{multirow}

\usepackage{tabularx}
\usepackage{fancyhdr}
\usepackage{lscape}
\usepackage{mdframed}
\usepackage{mathtools}
\usepackage{array}

\usepackage{mwe}
\usepackage[skip=2pt,font=scriptsize]{caption}
\captionsetup{labelfont=bf, format=hang, labelsep=period, labelformat = empty}
%labelformat  =simple
%labelformat = empty

\usepackage{appendixnumberbeamer} %for avoid counting appendix slides add \appendix
\usepackage{colortbl}
\usepackage{xcolor}

%\usetheme{Warsaw}
\usetheme{Madrid}
\usecolortheme{default}



%------------------------------------------------------------
%This block of code defines the information to appear in the
%Title page
\title[Improved Seed Adoption] %optional/
{Improved Seed Adoption and Climate Change: Hybrid Maize in Ethiopia}


\author[et al]{Barriga-Cabanillas,  Chiarella, Correa, Michuda}

% {Oscar Barriga-Cabanillas\inst{1} \\
% (with Travis Lybbert\inst{2})}

\institute[UC Davis] % (optional)
{
%  \inst{1}%
  Agricultural and Resource Economics\\
  UC Davis
 }

\date[Feb 2021] % (optional)
{CGIAR-SPIA \\
February 2021}


%End of title page configuration block
%------------------------------------------------------------



%------------------------------------------------------------
%The next block of commands puts the table of contents at the 
%beginning of each section and highlights the current section:

\AtBeginSection[]
{
  \begin{frame}
    \frametitle{Table of Contents}
    \tableofcontents[currentsection]
  \end{frame}
}
%------------------------------------------------------------


\begin{document}

\setbeamertemplate{enumerate items}[square]
\setbeamercolor{item projected}{bg=none,fg=black}

%\setbeamercolor{enumerate items}{〈key=value〉 list} to
%\setbeamercolor{item projected}{bg=none,fg=beamer@blendedblue}

\setbeamertemplate{itemize items}[circle]
\setbeamercolor{itemize item}{fg=black}
\setbeamercolor{itemize subitem}{fg=black}

%%% TIKZ STUFF
\tikzset{   
        every picture/.style={remember picture,baseline},
        every node/.style={anchor=base,align=center,outer sep=1.5pt},
        every path/.style={thick},
        }
\newcommand\marktopleft[1]{%
    \tikz[overlay,remember picture] 
        \node (marker-#1-a) at (-.3em,.3em) {};%
}
\newcommand\markbottomright[2]{%
    \tikz[overlay,remember picture] 
        \node (marker-#1-b) at (0em,0em) {};%
}
\tikzstyle{every picture}+=[remember picture] 
\tikzstyle{mybox} =[draw=black, very thick, rectangle, inner sep=10pt, inner ysep=20pt]
\tikzstyle{fancytitle} =[draw=black,fill=red, text=white]
%%%% END TIKZ STUFF



%The next statement creates the title page.
\frame{\titlepage}

% %---------------------------------------------------------
% %This block of code is for the table of contents after
% %the title page
% \begin{frame}
% \frametitle{Table of Contents}
% \tableofcontents
% \end{frame}
% %---------------------------------------------------------




%---------------------------------------------------------
\begin{frame}
\frametitle{Research question}

Food security is a major challenge: Ethiopia is no exception

\begin{itemize}
    \item Struggled to provide an adequate and reliable food supply
    \item Breakthroughs in maize germplasm ($\uparrow$ yields, $\uparrow$ drought tolerance)
\end{itemize}

However adoption of improved seed has remained a challenge

\begin{enumerate}
    \item Market integration
    \item Access to supplementary inputs
    \item Agro-ecological considerations
\end{enumerate}

\textbf{We study how heterogeneity masks the true impact of adoption}

\begin{enumerate}
    \item Adoption where  net benefits are the highest
    \item How adaptation strategies augment comparative advantage in adoption
    \item Drivers of dis-adoption of improved maize seed
\end{enumerate}



 
\end{frame}


%---------------------------------------------------------
\begin{frame}
\frametitle{Empirical Strategy I}

Correlated Random Coefficient Model and comparative advantage (Suri 2011; Tjernstr\"{o}m et al. 2020; Michler et al. 2019; Barriga et al. 2018)


\begin{itemize}
    \item Farmers face unobservable heterogeneous returns to adoption (comparative advantage)
\end{itemize}  

\begin{center}
$$
    y_{it}= \delta + \beta h_{it} + \theta_{i} + \phi\theta h_{it} + \tau_{i} + \epsilon_{it}
$$  
\end{center}

\begin{itemize}
    \item $y_{it}$: yields or profits for household i at time t
    \item $h$:  indicator for adoption
    \item $\tau$:  household’s absolute advantage
    \item $\theta$: household’s comparative advantage
    \item $\phi$: Gains from adopting, relative to a household’s comparative advantage. 
\end{itemize}

\end{frame}


%---------------------------------------------------------
\begin{frame}
\frametitle{Empirical Strategy II and Extensions}

Advantages:
\begin{enumerate}
    \item Identification does not rely on the existence of a valid instrumental variable (linear projection of observed adoption history)
    \item Disentangles a household’s absolute advantage from comparative advantage in adopting. 
\end{enumerate}

Extending the model:

\begin{itemize}
    \item Implementing a nonparametric panel identification in a novel GRC (Group Random Coefficients)
    \item Modeling and incorporating time-varying characteristics 
    \item Complement ESS data with precipitation (CHIRPS) and temperature (CPC) data
\end{itemize}  
\end{frame}


%---------------------------------------------------------


\end{document}
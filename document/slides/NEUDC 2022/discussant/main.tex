\documentclass{beamer}
\usepackage[utf8]{inputenc}
\usepackage{threeparttable}
%\usepackage{utopia} %font utopia imported

\usepackage{amsthm, booktabs, setspace, natbib, amsfonts,amssymb,epstopdf,textcomp,listings, xcolor}

\usepackage{adjustbox} % Shrink stuff
\usepackage{pgfpages}


\usepackage{tikz}
\usepackage{verbatim}
\setbeamertemplate{note page}{\pagecolor{yellow!5}\insertnote}
\usetikzlibrary{positioning}
\usetikzlibrary{snakes}
\usetikzlibrary{calc}
\usetikzlibrary{arrows}
\usetikzlibrary{decorations.markings}
\usetikzlibrary{shapes.misc}
\usetikzlibrary{matrix,shapes,arrows,fit,tikzmark}

\usepackage{amsmath}
\usepackage{mathpazo}
\usepackage{hyperref}
\usepackage{lipsum}
\usepackage{multimedia}
\usepackage{graphicx}
\usepackage{multirow}
\usepackage{dcolumn}
\usepackage{bbm}
\usepackage[space]{grffile}
\usepackage{tabularx}
\usepackage{subfig}
\usepackage{float}
\usepackage{multirow}

\usepackage{tabularx}
\usepackage{fancyhdr}
\usepackage{lscape}
\usepackage{mdframed}
\usepackage{mathtools}
\usepackage{array}

\usepackage{mwe}
\usepackage[skip=2pt,font=scriptsize]{caption}
\captionsetup{labelfont=bf, format=hang, labelsep=period, labelformat = empty}
%labelformat  =simple
%labelformat = empty

\usepackage{appendixnumberbeamer} %for avoid counting appendix slides add \appendix
\usepackage{colortbl}
\usepackage{xcolor}

%\usetheme{Warsaw}
\usetheme{Madrid}
\usecolortheme{default}
\usepackage{natbib}
\bibliographystyle{unsrtnat}


%------------------------------------------------------------
%This block of code defines the information to appear in the
%Title page
\title{Discussing: "Seeding the Seeds: Role of Social Structure in
Agricultural Technology Diffusion"}


\author[et al]{Alain de Janvry, Manaswini Rao, Elisabeth Sadoulet}


% \institute[UC Davis] % (optional)
% {
% %  \inst{1}%
%   Agricultural and Resource Economics\\
%   UC Davis
%  }

\date[November 2022] % (optional)
{Discussant:\\
Aleksandr Michuda}


%End of title page configuration block
%------------------------------------------------------------



%------------------------------------------------------------
%The next block of commands puts the table of contents at the 
%beginning of each section and highlights the current section:

\AtBeginSection[]
{
  \begin{frame}
    \frametitle{Table of Contents}
    \tableofcontents[currentsection]
  \end{frame}
}
%------------------------------------------------------------


\begin{document}

\setbeamertemplate{enumerate items}[square]
\setbeamercolor{item projected}{bg=none,fg=black}

%\setbeamercolor{enumerate items}{〈key=value〉 list} to
%\setbeamercolor{item projected}{bg=none,fg=beamer@blendedblue}

\setbeamertemplate{itemize items}[circle]
\setbeamercolor{itemize item}{fg=black}
\setbeamercolor{itemize subitem}{fg=black}

%%% TIKZ STUFF
\tikzset{   
        every picture/.style={remember picture,baseline},
        every node/.style={anchor=base,align=center,outer sep=1.5pt},
        every path/.style={thick},
        }
\newcommand\marktopleft[1]{%
    \tikz[overlay,remember picture] 
        \node (marker-#1-a) at (-.3em,.3em) {};%
}
\newcommand\markbottomright[2]{%
    \tikz[overlay,remember picture] 
        \node (marker-#1-b) at (0em,0em) {};%
}
\tikzstyle{every picture}+=[remember picture] 
\tikzstyle{mybox} =[draw=black, very thick, rectangle, inner sep=10pt, inner ysep=20pt]
\tikzstyle{fancytitle} =[draw=black,fill=red, text=white]
%%%% END TIKZ STUFF



%The next statement creates the title page.
\frame{\titlepage}

% %---------------------------------------------------------
% %This block of code is for the table of contents after
% %the title page
% \begin{frame}
% \frametitle{Table of Contents}
% \tableofcontents
% \end{frame}
% %---------------------------------------------------------




%---------------------------------------------------------
\begin{frame}{Short Summary}
    \begin{itemize}
        \item Returning to the site of two-staged randomized experiment in Emerick et al. (2016)
        \item Introduction of flood resistant paddy variety 
        \item Evaluating diffusion of seed from treatment to non-treatment households
        \item Dynamic treatment effects mask heterogeneity found across jati (sub-castes) in regions
        \item Less village-level fractionalization of jati leads to higher levels of diffusion and adoption by non-treatment households
    \end{itemize}
\end{frame}

\begin{frame}{Highlights}
    \begin{itemize}
        \item Compelling story; just as important to look at spillovers as for treatment, itself
        \item Merges two interesting literatures: seed tech. adoption and economics of Indian caste
        \item has clear policy implications for future policy 
        interventions 
        \item In particular, social hierarchy results are very interesting!
    \end{itemize}    
\end{frame}

\begin{frame}{Feedback}
    \begin{itemize}
        \item Regression of Flood x Treatment unclear for untreated households
            \begin{itemize}
                \item Being in more flood-prone areas will increase rate of adoption?
                \item perhaps using lagged treatment would do better?
            \end{itemize}
        \item Why use an indicator for ELF, and not the continuous variable?
        \item In many cases, for a particular year, adoption and area cultivated increases with similar jati rank, or having someone from your jati receive the mini-kit, but the seed source outcome stays insignificant or is only significant in 2015
            \begin{itemize}
                \item The argument in the paper is that this diffusion happens through one's network, however, the lack of results from seed source makes me want a story about mechanisms. How are these seeds being used if not through seed exchange?
                \item If seed is readily available, is this story more about ``information'' exchange than the physical seed?
            \end{itemize}
    \end{itemize}
\end{frame}

\begin{frame}{Feedback}

\begin{itemize}
    \item How often does a treated village also get treated with a flood?
    \item In some cases, I'm inclined to think that the paper might be have power challenges, and whether it might be useful to have a continuous flood measure? Villages that might not have full-on floods, but that are close and observe floods might also be more likely to adopt
\end{itemize}

\textbf{Minor Feedback}
    \begin{itemize}
        \item Why not make all tables into figures as you have for other regressions?
    \end{itemize}
\end{frame}


\bibliography{references}

\end{document}